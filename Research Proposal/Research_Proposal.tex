\documentclass[]{elsarticle} %review=doublespace preprint=single 5p=2 column
%%% Begin My package additions %%%%%%%%%%%%%%%%%%%
\usepackage[hyphens]{url}
\usepackage{lineno} % add
\providecommand{\tightlist}{%
  \setlength{\itemsep}{0pt}\setlength{\parskip}{0pt}}

\bibliographystyle{elsarticle-harv}
\biboptions{sort&compress} % For natbib
\usepackage{graphicx}
\usepackage{booktabs} % book-quality tables
%% Redefines the elsarticle footer
%\makeatletter
%\def\ps@pprintTitle{%
% \let\@oddhead\@empty
% \let\@evenhead\@empty
% \def\@oddfoot{\it \hfill\today}%
% \let\@evenfoot\@oddfoot}
%\makeatother

% A modified page layout
\textwidth 6.75in
\oddsidemargin -0.15in
\evensidemargin -0.15in
\textheight 9in
\topmargin -0.5in
%%%%%%%%%%%%%%%% end my additions to header

\usepackage[T1]{fontenc}
\usepackage{lmodern}
\usepackage{amssymb,amsmath}
\usepackage{ifxetex,ifluatex}
\usepackage{fixltx2e} % provides \textsubscript
% use upquote if available, for straight quotes in verbatim environments
\IfFileExists{upquote.sty}{\usepackage{upquote}}{}
\ifnum 0\ifxetex 1\fi\ifluatex 1\fi=0 % if pdftex
  \usepackage[utf8]{inputenc}
\else % if luatex or xelatex
  \usepackage{fontspec}
  \ifxetex
    \usepackage{xltxtra,xunicode}
  \fi
  \defaultfontfeatures{Mapping=tex-text,Scale=MatchLowercase}
  \newcommand{\euro}{€}
\fi
% use microtype if available
\IfFileExists{microtype.sty}{\usepackage{microtype}}{}
\ifxetex
  \usepackage[setpagesize=false, % page size defined by xetex
              unicode=false, % unicode breaks when used with xetex
              xetex]{hyperref}
\else
  \usepackage[unicode=true]{hyperref}
\fi
\hypersetup{breaklinks=true,
            bookmarks=true,
            pdfauthor={},
            pdftitle={Research proposal for Claridge Canal},
            colorlinks=true,
            urlcolor=blue,
            linkcolor=magenta,
            pdfborder={0 0 0}}
\urlstyle{same}  % don't use monospace font for urls
\setlength{\parindent}{0pt}
\setlength{\parskip}{6pt plus 2pt minus 1pt}
\setlength{\emergencystretch}{3em}  % prevent overfull lines
\setcounter{secnumdepth}{0}
% Pandoc toggle for numbering sections (defaults to be off)
\setcounter{secnumdepth}{0}
% Pandoc header


\usepackage[nomarkers]{endfloat}

\begin{document}
\begin{frontmatter}

  \title{Research proposal for Claridge Canal}
    \author[NC State University]{Qianyu Hang}
   \ead{qhang@ncsu.edu} 
  
    \author[NC State University]{François Birgand\corref{c1}}
   \ead{birgand@ncsu.edu} 
   \cortext[c1]{François Birgand}
      \address[NC State University]{Biological and Agricultural Engineering, Capus Box 7625, Raleigh, NC,
27695}
  
  \begin{abstract}
  A growing number of stream restoration projects have been undertaken to
  counteract the adverse effects of stream degradation worldwide.
  Nevertheless, controversies on the real impacts of stream restoration
  still exist. The objective of this paper is to review the current
  knowledge base on the impacts of stream restoration in water quality,
  hydromorphology and fish/invertebrate community. Based on the database
  developed in this study, 67\% of projects indicated a positive effect of
  nitrogen reduction following restoration. Hydromorphological
  improvements were identified almost by all hydromorphological
  restoration efforts. As fish/invertebrate community require a long
  time-frame to recover, reaching common ground can therefore be difficult
  due to different time-frames applied by researchers. Given that
  restoration studies are usually site- and approach-specific, it is
  impossible to predict every outcome of a stream restoration project by
  simply extrapolating the results from one single study. However, finding
  some robust indicators for which could show the uniform trend following
  restoration is helpful for evaluating the effectiveness of stream
  restoration projects. Several future research needs should be
  highlighted: 1) implement high-resolution data analysis, in particular
  for water quality; 2) choose robust indicators for restoration project
  evaluation; and 3) avoid comparison using nearby reference streams.
  \end{abstract}
  
 \end{frontmatter}

\hypertarget{references}{%
\section*{References}\label{references}}
\addcontentsline{toc}{section}{References}

\end{document}


